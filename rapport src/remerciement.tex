


%%%%%%%%%%%%%%%%
 
\chapter*{Remerciements}\addcontentsline{toc}{chapter}{Remerciements}

\hspace*{10pt}Un mémoire est un travail de recherche long et ardu. Malgré
que ce soit un travail en solitaire, il aurait été difficile d'occulter de l'appui de
nombreuses personnes d'un point de vue académique et moral, à qui je tiens à exprimer toute ma reconnaissance. Je tiens à exprimer mes remerciements et ma vive gratitude à : 
\begin{itemize}
		\item[\textbullet]	Pr. MOTAPON Ousmanou qui me fait l'honneur de présider ce jury de mémoire ;
		\item[\textbullet]  Dr. Tchangnwa  Nya Fridolin pour avoir accepté d'évaluer ce travail ;
		\item[\textbullet]  Prof. Dr.-Ing. habil. KOLYANG  qui a accepté d'être le rapporteur de ce mémoire, je le remercie tout particulièrement pour l'attention et le temps qu'il y a consacré ;
		\item[\textbullet]	Ma mère LAKGUEM Suzanne qui a cru en moi et m'a soutenu tout le long de mes études ;
		\item[\textbullet]  mes frères et sœurs, NIMPA Collins, LEBOU Ghislain, DASSE Odosine pour leurs soutiens financier et morale pendant toutes la durée de ma formation ;
		\item[\textbullet]	Mlle. LOBAWO Adeline Zoustel pour son amour, son soutien moral et ses encouragements ;
		\item[\textbullet]  la famille PASSO de Maroua pour m'avoir accueilli comme un fils. 
		\item[\textbullet]  tous les enseignants qui m'ont suivi tout au long de cette formation.
		\end{itemize}
	
	 Je remercie particulièrement ma famille pour le soutien qu'elle m'a apporté et la
motivation qu'elle a su me donner lorsque j'en avais le plus besoin.\\

 J'exprime  également  toute  ma  sympathie  et  ma  gratitude à mes collègues de stage, MOUAFO Joseph, ZANGUE Josue et les autres pour l'ambiance agréable qu'ils ont su créer.\\ 



