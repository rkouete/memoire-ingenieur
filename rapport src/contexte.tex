


%%%%%%%%%%%%%%%%
 
\chapter{CONTEXTE ET PROBLÉMATIQUE}

\hspace*{10pt} Dans ce chapitre, nous présentons d'abord brièvement la structure dans laquelle
nous avons effectué notre stage de fin d'étude, le contexte, puis la problématique sur laquelle notre travail est basé. Ensuite viendront les objectifs fixés ainsi que la méthodologie adoptée pour parvenir à bout de ces objectifs. 
\section{Présentation de la structure de stage}
Notre stage de fin de formation s'est déroulé du 03 mars au 10 juillet 2016 au laboratoire informatique de l'Institut Supérieur de Sahel (ISS) de l'Université de Maroua. L'ISS est l'une des grandes écoles de l'Université de Maroua. Ouvert en 2008, sa principale mission est de former de jeunes étudiants dans les domaines de :
\begin{itemize}
	\item  [\textbullet] Agriculture, Élevage et Produits Dérivés (AGEPD) ;
	\item  [\textbullet] Beaux-arts et science du patrimoine (BEARSPA) ;
	\item  [\textbullet] Climatologie, Hydrologie et Pédologie 
	\item  [\textbullet] Énergies renouvelables (ENREN) ;
	\item  [\textbullet] Hydraulique et Maîtrise des eaux (HYMAE) ;
	\item  [\textbullet] Informatique et Télécommunications (INFOTEL) ;
	\item  [\textbullet] Génie du Textile et Cuir (GTC) ;
	\item  [\textbullet] Sciences Environnementales (SCIENV) ;
	\item  [\textbullet] Sciences Sociales pour le Développement (SCISOD) ;
	\item  [\textbullet] Traitement des Matériaux, Architecture et Habitat (TRAMARH).
\end{itemize}
Dans chaque département, les formations sont offertes en deux cycles : le cycle ingénieur des travaux dont la durée est de trois ans et le cycle ingénieur de conception pour une durée de cinq ans. A l'exception du département de Climatologie qui offre en plus un cycle BTS.

Le laboratoire informatique de l'ISS et le département d'INFOTEL partagent le même bâtiment au campus de pitoré (IRAD) et sont tous dirigés par Dr. VIDEME BOSSOU Olivier. 

\section{Contexte}

De nos jours, la fraude ne cesse d'augmenter dans la société. Certains
utilisateurs, (imposteurs) sont capables de falsifier leur identité avec une
facilité remarquable. Ceci est dû au fait que les systèmes d'authentification les
plus utilisés sur le marché sont basés sur la solution conventionnelle: \og login and
password\fg{} ou accès avec mot de passe, souvent associée à une carte ID
contenant l'information sur d'identité de son possesseur. Cependant, les
utilisateurs ne sont pas tout à fait satisfaits de ces cartes ID pour les raisons
suivantes:
\begin{itemize}
	\item les cartes ID basées sur les mots de passe ne sont pas fiables,
	\item les cartes ID peuvent être perdues, oubliées ou mal placées,
	\item le mot de passe risque d'être oublié ou compromis.
\end{itemize}
Les utilisateurs ne se sentent pas suffisamment bien sécurisés en effectuant des transactions faisant appel à de telles techniques d'authentification.  

Une étude \footnote{Cette étude intitulée : « Businesses Should Begin Preparing for the Death of the Password » a été réalisée par OnePoll.com pour Gigya, en février 2016, auprès de 2 000 adultes américains et 2 000 adultes britanniques entre 18 et 69 ans ayant accès à Internet et étant titulaires de comptes en ligne.} menée auprès de 4 000 consommateurs aux Etats-Unis et au Royaume-Uni \cite{BIOM}a montré que 52\% des consommateurs  choisiraient si on leur proposait  une autre option que celle de la traditionnelle \og login/mot de passe\fg{} pour s'inscrire sur un site web ou sur une application. Selon \cite{BIOM}, 25\% ont été victime d'une compromission de leur compte au cours des 12 derniers mois. Ce taux monte à 35\% pour la génération Y \footnote{La génération Y regroupe des personnes nées approximativement entre le début des années 1980 et le milieu des années 1990.}, car souvent moins rigoureux dans la détermination de leur mot de passe. Certaines entreprises complexifient leur politique de génération de mot de passe, mais cela alourdi les procédure d'inscription et décourage certains utilisateurs.

Les entreprises doivent donc moderniser leurs méthodes d'authentification sous peine d'en subir les conséquences. La plupart se tournent vers la biométrie. 
\section{Problématique}

Face aux différents problèmes que posent les méthodes traditionnelles, on se demande comment construire un système d'authentification physique en temps réel et qui donne le plus de confort aux utilisateurs tout en garantissant un accès hautement sécurisé? 

\section{Objectifs}
Différents types de systèmes sont utilisés pour l'identification/authentification en temps réel. Les  plus  populaires  sont  basés  sur  le  visage, l'empreinte digital et l'iris. Notre objectif est de proposer un système biométrique basé sur la reconnaissance faciale respectant des contraintes  temps  réels.

\section{Méthodologie}
Pour résoudre ce problème, nous allons dans un premier temps répertorier les différents techniques et algorithmes de reconnaissance faciale, ensuite étudier leur performances et enfin implémenter le système qui convient le mieux (c'est-à-dire celui qui respecte les conditions temps réel et qui admet un taux acceptable de reconnaissance), puis procéder à des tests de validation. 
 