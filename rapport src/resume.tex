\chapter*{Résumé}\addcontentsline{toc}{chapter}{Résumé}

 La reconnaissance faciale est une technologie biométrique non intrusive et sans contact. Ces caractéristiques font d'elle un domaine de recherche aux publications abondantes. Bien que de nombreuses méthodes et techniques de reconnaissance de visage aient été proposées, elle reste encore un problème difficile, car les facteurs externes comme la variation d'illumination, la variation de pose et d'expressions faciales ne sont pas toujours traités de manière satisfaisante par la plupart des algorithmes. De nos jours les systèmes d'identification et de vérification biométrique constituent une alternative de taille à l'authentification traditionnelle.

Ce mémoire traite de l'identification automatique de visage sur une photo ou dans un flux vidéo. Deux méthodes de reconnaissance sont présentées : la méthode \textit{eigenface} qui utilise l'analyse en composantes principales et la méthode des "Local Binary Patterns". Ces deux algorithmes ont été testés sur les bases de données ORL et YALE et les résultats obtenus montrent que la technique LBP donne de meilleurs résultats.
 
\vspace{20pt}
\textbf{Mots clés :} Reconnaissance faciale, ACP, Eigenface, LBP