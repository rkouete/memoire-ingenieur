\chapter*{CONCLUSION ET PERSPECTIVES}\addcontentsline{toc}{chapter}{CONCLUSION ET PERSPECTIVES}

Dans ce travail, nous nous sommes intéressé au problème de reconnaissance faciale. Nous avons mis sur pied un système de reconnaissance faciale basé sur les algorithmes Eigenface et LBP qui représentent deux des algorithmes de reconnaissance les plus utilisés. La méthode de détection de visage utilisé dans les deux cas est l'algorithme de Viola et Jones \cite{VIO}. C'est une méthode largement reconnue comme fonctionnant en temps réel et avec de très bon résultats ( plus de 98\% pour des visages présentés de face, plus de 94\% pour des visages faisant moins de 20\degres. Nous avons trouver ces résultats assez convaincant pour l'utilisation de l'algorithme de Viola et Jones dans notre application.

 L'algorithme Eigenface est basé sur la technique mathématique ACP (Analyse en composantes principales) qui est un moyen de simplifier en ensemble de données en réduisant sa dimension. Elle  est  utilisée  pour  représenter efficacement les images  de visages, qui peuvent être approximativement  reconstruites à partir d'un petit ensemble de poids et d'une image de visage standard. L'algorithme LBP quant à elle consiste à construire l'histogramme d'une image en utilisant le code LBP de tous ces pixels préalablement calculé. L'histogramme LBP permet de construire un vecteur de  caractéristiques représentant l'image faciale.

Les systèmes de reconnaissance sont pour la plupart influencés par les problèmes d'illumination, de variations de pose et d'éclairage. Nous avons proposé dans ce mémoire quelques solutions à ce problème. Mais malgré les progrès réalisés, les variations de pose restent un sérieux défis à relever dans la reconnaissance dans des environnements extérieurs, dont suscite des efforts de la part des chercheurs. Néanmoins l'ACP est une approche efficace et simple pour remédier à ce problème.

En guise de perspectives, ce travail peut être étendu dans un premier temps par l'amélioration de ses capacités à résister aux variations d'environnements. Et dans un second temps par 
la conception et la mise en œuvre d'un système de reconnaissance à performances assez hautes (utilisant par exemple la reconnaissance en 3D), pouvant servir à pister un individu à l'aide d'un réseau de caméras. 
