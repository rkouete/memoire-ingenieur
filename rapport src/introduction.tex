


%%%%%%%%%%%%%%%%
 
\chapter*{INTRODUCTION}\addcontentsline{toc}{chapter}{INTRODUCTION}
\renewcommand{\thepage}{\arabic{page}}
\setcounter{page}{1}
%\thispagestyle{empty}
De nos jours, l'insécurité devient de plus en plus présente dans divers secteurs. Les secteurs les plus visés sont l'e-commerce, les opérations bancaires, militaires. Les moyens informatiques sont proposés pour lutter contre cette tendance parmi lesquels la méthode de contrôle d'accès. Le contrôle d'accès logique s'effectue en trois étapes : l'identification, l'authentification et l'autorisation. S'identifier, c'est se faire connaître du système tandis que s'authentifier c'est prouver son identité au système. Quant à l'autorisation, elle donne le droit d'effectuer une opération précise. Il existe deux manières d'authentifier un individu. La première méthode est basée sur \og ce que l'on connaît\fg{} de l'individu comme la connaissance d'un couple identifiant/mot de passe. La deuxième méthode est basée sur \og ce que l'on possède \fg{} par exemple la possession d'une pièce d'identité, d'un badge, d'une clé, etc. Ces deux méthodes peuvent être utilisées de manière complémentaire dans le but d'accroître la sécurité mais présente chacune des faiblesses. Par exemple un mot de passe peut être oublié, volé ou deviné par autrui. De même une pièce d'identité (ou clé) peut être perdue, volée par une personne mal intentionnée.\\

\hspace*{10pt}La biométrie est une technique globale visant à établir l'identité d'une personne en mesurant une de ses caractéristiques physiques. Roethenbaugh \citep{Roe12} définit la biométrie de la façon suivante : \og \textit{La  biométrie  s'applique à des particularités ou des caractères humains uniques en leur genre et mesurables, permettant de reconnaître ou de vérifier automatiquement l'identité}\fg{}. Les caractéristiques biométriques constituent ainsi une solution alternative aux deux méthodes d'authentification évoquées précédemment. Elles présentent l'avantage d'être uniques et présentes chez toutes les personnes à identifier.\\

\hspace*{10pt}Il existe différents types de technologies biométriques à savoir la reconnaissance faciale, la reconnaissance d'empreintes digitales, l'identification de la géométrie des doigts et de la main, l'identification de l'iris, l'identification de la voix, l'identification de la signature, etc.\\

\hspace*{10pt}La reconnaissance faciale est une aptitude qui relie l'apparence d'une personne à son identité. La capacité pour l'identification de visage est très importante pendant notre vie sociale. Chacun de nous identifie tout le long de la journée différents visages. A la rencontre d'une nouvelle personne, notre cerveau recherche dans notre mémoire si cette personne est répertoriée ou non. La reconnaissance faciale permet ainsi de construire une relation à long terme où les individus finissent par se connaître. \\

\thispagestyle{empty}

\hspace*{10pt}Grâce à la puissance croissante de l'informatique et aux prix décroissants des ordinateurs, des applications informatiques sont appliquées populairement dans notre vie quotidienne. Dans le domaine de la vision par ordinateur, la reconnaissance de visage a attiré beaucoup l'attention des chercheurs. Malgré que plusieurs idées intéressantes et utiles soient déjà proposées, ce problème n'est pas souvent traité complètement. Et il est encore un problème ouvert qui demande plus d'étude profonde. En fait, ce problème n'est pas simplifié.

La reconnaissance faciale a de nombreux champs d'applications : télésurveillance, vérification d'identité pour le déverrouillage d'ordinateur ou la traque de suspects à travers les enregistrements vidéo, reconnaissance des expressions faciales, etc. 

Nous avons choisi d'articuler notre étude autour de cinq chapitres principaux. Le chapitre 1 présente le contexte de notre travail ainsi que la problématique sur laquelle il se base. Le deuxième chapitre est consacré aux généralités sur les systèmes biométriques, il décrit quelques systèmes biométriques ainsi que leur principe de fonctionnement. Ensuite la place de la reconnaissance faciale parmi les autres techniques biométriques est analysée. Enfin nous présentons les difficultés rencontrées dans les systèmes de reconnaissance faciale. Dans le chapitre 3, nous présentons un état de l'art sur les techniques et méthodes de détection et de reconnaissance de visages. Dans le chapitre 4, nous expliquons comment nous avons développé un système de reconnaissance faciale utilisant les algorithmes eigenface et LBP. Enfin dans le cinquième et dernier chapitre, nous présentons les résultats et tests de performance que nous analyserons.





\nocite{MAL}

